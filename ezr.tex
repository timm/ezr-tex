%%% LaTeX Template: Two column article
%%%
%%% Source: http://www.howtotex.com/
%%% Feel free to distribute this template, but please keep to   to http://www.howtotex.com/ here.
%%% Date: February 2011

%%% Preamble
\documentclass[landscape,	DIV=calc,%
							paper=letter,%
							fontsize=10pt,%
							twocolumn]{scrartcl}	 					% KOMA-article class

               \setlength{\columnsep}{20px}
\bibliographystyle{plain}
\makeatletter
\newcommand\notsotiny{\@setfontsize\notsotiny{6.5}{7.5}}
\makeatother
\usepackage{lipsum}													% Package to create dummy text

\usepackage[margin=.5in,footskip=0.25in]{geometry}

\usepackage[english]{babel}										% English language/hyphenation
\usepackage[protrusion=true,expansion=true]{microtype}				% Better typography
\usepackage{amsmath,amsfonts,amsthm}					% Math packages
\usepackage[pdftex]{graphicx}									% Enable pdflatex
\usepackage[svgnames,usenames,dvipsnames]{xcolor}

\usepackage[sc]{mathpazo}          % Palatino for roman (\rm)
\usepackage[scaled=0.95]{helvet}   % Helvetica for sans serif (\sf)
\usepackage{courier}               % Courier for typewriter (\tt)

\usepackage{sourcecodepro}
\usepackage[T1]{fontenc} 


\usepackage{listings} 

\definecolor{codegreen}{rgb}{0,0.9,0}
\definecolor{codegray}{rgb}{0.55,0.55,0.55}
\definecolor{codepurple}{rgb}{0.58,0,0.82}
\definecolor{backcolour}{rgb}{0.95,0.95,0.92}

\definecolor{commentgreen}{RGB}{2,112,10}
\definecolor{eminence}{RGB}{108,48,130}
\definecolor{weborange}{RGB}{255,165,0}
\definecolor{frenchplum}{RGB}{129,20,83}
\definecolor{CustomDarkRed}{RGB}{175, 0, 0} 
 

\lstdefinestyle{mystyle}{
    xleftmargin=2em,
    language=python,
    %backgroundcolor=\color{codegray},   
    commentstyle=\color{codegray},
    keywordstyle=\bfseries\color{CustomDarkRed},
    numberstyle=\tiny\color{codegray},
    stringstyle=\color{blue},
    basicstyle=\ttfamily\scriptsize,
    breakatwhitespace=false,         
    breaklines=true,                 
    captionpos=b,                    
    keepspaces=true,                 
    numbers=left,                    
    numbersep=5pt,                  
    showspaces=false,                
    showstringspaces=false,
    showtabs=true,                  
    tabsize=2,
    mathescape=true,
    frame=tb, 
}

\lstset{style=mystyle}
% Enabling colors by their 'svgnames'
\usepackage[hang, small,labelfont=bf,up,textfont=it,up]{caption}	% Custom captions under/above floats
\usepackage{epstopdf}												% Converts .eps to .pdf
\usepackage{subfig}													% Subfigures
\usepackage{booktabs}												% Nicer tables
\usepackage{fix-cm}													% Custom fontsizes



%%% Custom sectioning (sectsty package)
\usepackage{sectsty}													% Custom sectioning (see below)
\allsectionsfont{%															% Change font of al section commands
	\usefont{OT1}{phv}{b}{n}%										% bch-b-n: CharterBT-Bold font
	}

\sectionfont{%																% Change font of \section command
	\usefont{OT1}{phv}{b}{n}%										% bch-b-n: CharterBT-Bold font
	}



%%% Headers and footers
\usepackage{fancyhdr}												% Needed to define custom headers/footers
	\pagestyle{fancy}														% Enabling the custom headers/footers
\usepackage{lastpage}	

% Header (empty)
\lhead{}
\chead{}
\rhead{}
% Footer (you may change this to your own needs)
\lfoot{\footnotesize \texttt{Easier AI} \textbullet ~a programmer's guide}
\cfoot{}
\rfoot{\footnotesize page \thepage\ of \pageref{LastPage}}	% "Page 1 of 2"
\renewcommand{\headrulewidth}{0.0pt}
\renewcommand{\footrulewidth}{0.4pt}



%%% Creating an initial of the very first character of the content
\usepackage{lettrine}
\newcommand{\initial}[1]{%
     \lettrine[lines=3,lhang=0.3,nindent=0em]{
     				\color{DarkGoldenrod}
     				{\textsf{#1}}}{}}


\usepackage{tikz}
\definecolor{alizarin}{rgb}{0.82, 0.1, 0.26}

\newcommand*\circled[1]{\tikz[baseline=(char.base)]{
            \node[minimum width=1pt, shape=circle,fill=black,inner sep=1pt] (char) {{\footnotesize \textcolor{white}{#1}}};}}

            \newcommand{\A}{\circled{a}}
            \newcommand{\two}{\circled{b}}
            \newcommand{\three}{\circled{c}}
            \newcommand{\four}{\circled{d}}
            \newcommand{\five}{\circled{e}}
            \newcommand{\six}{\circled{f}}
            \newcommand{\seven}{\circled{g}}
            \newcommand{\eight}{\circled{h}}
            \newcommand{\nine}{\circled{i}}
            \newcommand{\ten}{\circled{j}}
            \newcommand{\eleven}{\circled{k}}
            \newcommand{\twelve}{\circled{l}}
            \newcommand{\thirteen}{\circled{m}}
            \newcommand{\fourteen}{\circled{n}}
            \newcommand{\fifteen}{\circled{o}}
            \newcommand{\sixteen}{\circled{p}}
            \newcommand{\seventeen}{\circled{q}}
            \newcommand{\eighteen}{\circled{r}}
            \newcommand{\nineteen}{\circled{s}}
            \newcommand{\twenty}{\circled{t}}



%%% Title, author and date metadata
\usepackage{titling}															% For custom titles

\newcommand{\HorRule}{\color{DarkGoldenrod}%			% Creating a horizontal rule
									  	\rule{\linewidth}{1pt}%
										}
%%begin novalidate
\pretitle{\vspace{-70pt} \begin{flushleft} \HorRule 
				\fontsize{50}{50} \usefont{OT1}{phv}{b}{n} \color{DarkRed} \selectfont 
				}
    \usepackage[hidelinks]{hyperref}
\title{Easier AI }					% Title of your article goes here
\posttitle{\\\vspace{3mm}
\Large A programmer's guide to building simpler, smarter, faster, more
flexible and understandable analytics.
\par\end{flushleft}\vskip 0.5em}

\preauthor{\begin{flushleft}
					\large \lineskip 0.5em \usefont{OT1}{phv}{b}{sl} \color{DarkRed}}
     
\author{Tim Menzies and the EZR gang }											% Author name goes here
\postauthor{\footnotesize \usefont{OT1}{phv}{m}{sl} \color{Black} 
          North Carolina State University	\\~\\\today\\
      \url{https://doi.org/10.5281/zenodo.11183059}\\~\\
      Package: \url{https://pypi.org/project/ezr/0.1.0/}\\
      Source: \url{http://github.com/timm/ezr}\\
      Latex: \url{http://github.com/timm/ezr-tex}\\
     {\textcopyright} 2024 by Tim Menzies and the EZR gang is licensed under \textcolor{DarkRed}{Creative Commons Attribution-ShareAlike 4.0 International}
    \LARGE ~\ccLogo  
~\ccAttribution  
~\ccShareAlike  \par\end{flushleft}\vspace{-4mm}\HorRule}
%%end novalidate
\date{}	
\usepackage{ccicons}% No date
\usepackage{enumitem}
\setlist[itemize]{noitemsep}


%%% Begin document
\begin{document}
\maketitle
\section*{About}
\thispagestyle{fancy} 			% Enabling the custom headers/footers for the first page 
% The first character should be within \initial{}
\initial{A}\textbf{nalytics is the process of
extracting high-quality  insights from large quantities
of data. We show that a very
simple and very fast
AI analytics toolkit can  be built by reorganizing, combining, and simplifying
many seemingly different parts of that toolkit. The result is
less complexity, more efficiency, increased analytical
power, all from
methods requiring fewer data samples. This ``data-lite'' method
allows for easier verification and understanding of results. We
highlights the benefits of using incremental methods  in building
models that can provide valuable insights with minimal data.~\\~\\ 
This  work can be viewed as a (polite) protest
against  the prevailing preference for complex solutions
in the industry,  suggesting that simplicity could offer more
practical and appreciable benefits but is often overlooked due to
commercial interests. We call for, when possible,
a shift towards simplicity
in analytics, making it faster, smarter, and more flexible, to
better serve practical needs and enhance comprehensibility.}

\subsection*{Audience}
Programemrs, teachers of grad classes. XXX\newpage

\tableofcontents


\clearpage \section{Introduction}

Suppose we want to use data to make policies-- about what to do,
what to avoid, what to do better, etc etc. How to do that?

This process is called {\em analytics}, i.e. the reduction of large
amounts of low-quality data into tiny high-quality statements. Think of it like
``finding the diamonds in the dust``. 

At first glance, an analytics toolkit needs many functions.   For example,
in one survey of managers at   Microsoft, 
researchers found nine kinds  of analytics functions~\cite{buse2012information}.
As shown in the following table, those  functions include regression, topic analysis, anomaly detection, what-if analysis,
etc:


\includegraphics[width=\linewidth]{Buse.png} 

Software engineers have a superpower that lets them simplify long lists of functions
(like the above). That superpower is called {\em refactoring}, i.e. 
restructuring the source code  so as to improve operation.
This document applies  refactoring to analytics. It will be seen that, under
the hood, many
analytics tasks share a similar set of underlying classes.  This means that
once we code one analytics
function, then we can quickly 
code up many more. 

For example, suppose we code a  DATA class that stores rows of data.
This class:
\begin{itemize}
    \item
Summarizes the columns of that data  in   NUMeric and  SYMbolic
classes
(one for each column);
\item Knows how to report the expected middle values of NUMs and SYMs
    (which is the mean or mode  for NUMs or SYMs);
\item Knows how to report the
diversity about that middle value (which is standard deviation or entropy for
NUMs or SYMs).
\end{itemize}
This DATA class offers most of the code needed to implement  clustering  and  classification:
\begin{itemize}
    \item
        A k-means clusterer picks centroids and random, then labels each row according to 
        its nearest centroid. Those centroids are then moved to the middle of all rows with the same label and
        the process repeats. If all the rows with the same label are stored in  a DATA class, then ``moving the centroids''
        just means asking our NUMs and SYMs for their middle values.
\item A Naive Bayes classifier keeps separate statistics for all the rows with the same classification.
    If each class is implemented by a DATA class, then all those statistics can be collected just by using the DATA code.
\end{itemize} 
Better yet, 
once we have a clusterer and a classifier.


 
the data (e.g. during a "what-if" query).  But these days, I can
do the same analysis with 30 samples, or 
less\footnote{Using semi-supervised multi-objective optimization via
sequential model optimization (which is all described, later in
this document).} 
This means
if someone wants to check my conclusions, they only need to review
a few dozen samples.  Such a review was impossible using prior
methods since the reasoning was so complicated.


Why can I do things so easily? Well,  based on three decades of work
on analytics~\cite{menzies1988combining} (which includes the work of 20 Ph.D. students,
hundreds of research papers and millions of dollars in research
funding) I say:

- When building models, there are incremental methods that can find
models after very few samples.
- This is because the main message of most models is contained in
just a few variables~\cite{menzies1988combining}.

I'm not the first to say these things\footnote{
From Wikipedia: The manifold hypothesis posits that many
high-dimensional data sets that occur in the real world actually
lie along low-dimensional latent manifolds inside that high-dimensional
space. As a consequence of the manifold hypothesis, many data sets
that appear to initially require many variables to describe, can
actually be described by a comparatively small number of variables,
likened to the local coordinate system of the underlying manifold.}.
So it is a little
strange that someone else has not offer something like this simpler
synthesis. But maybe our culture prefers complex solutions:

\begin{quote}{\em Simplicity is a great virtue but it requires hard work to achieve
it and education to appreciate it. And to make matters worse:
complexity sells better.}\newline
-- Edsger W. Dijkstra
\end{quote}

By making things harder than they need to be, companies can motivate
the sale  of intricate tools to clients who wished there was a
simpler way. Well, maybe there is.


\begin{table}[!t]{\linewidth}
\begin{lstlisting}[caption=Python example]
import re,ast
from typing import Any,Iterable,Callable
from fileinput import FileInput as file_or_stdin
#---------- ---------- ---------- ---------- ---------- ---------- ----------
def coerce(s:str) -> Any:
  "s is a int,float,bool, or a string"
  try: return ast.literal_eval(s) # 
  except Exception:  return s

def csv(file=None) -> Iterable[Row]:  $\A$
  "read from file or standard input"
  with file_or_stdin(file) as src: 
    for line in src:
      line = re.sub(r'([\n\t"\’ ]|#.*)', '', line) # no comments,white space
      if line: yield [coerce(s.strip()) for s in line.split(",")]
#---------- ---------- ---------- ---------- ---------- ---------- ----------
class COLS(OBJ): 
  """Turns a list of names into NUMs and SYMs columns. All columns are held 
  in i.all.  For convenience sake, some are also help in i.x,i.y 
  (for independent, dependent cols) as well as i.klass (for the klass goal, 
  if it exists)."""
  def __init__(i, names: list[str]): 
    i.x, i.y, i.all, i.names, i.klass = [], [], [], names, None
    for at,txt in enumerate(names):
      a,z = txt[0], txt[-1] % first and last letter
      col = (NUM if a.isupper() else SYM)(at=at,txt=txt)
      i.all.append(col)
      if z != "X": # if not ignoring, maybe make then klass,x, or y
        (i.y if z in "!+-" else i.x).append(col)
        if z == "!": i.klass= col

  def add(i,row: Row) -> Row: 
    "summarize a row into the NUMs and SYMs"
    [col.add(row[col.at]) for col in i.all if row[col.at] != "?"]
    return row
\end{lstlisting}
\end{table}

I'm not the first to say these things\footnote{
From Wikipedia: The manifold hypothesis posits that many
high-dimensional data sets that occur in the real world actually
lie along low-dimensional latent manifolds inside that high-dimensional
space. As a consequence of the manifold hypothesis, many data sets
that appear to initially require many variables to describe, can
actually be described by a comparatively small number of variables,
likened to the local coordinate system of the underlying manifold.}.
So it is a little
strange that someone else has not offer something like this simpler
synthesis. But maybe our culture prefers complex solutions:

\begin{quote}{\em Simplicity is a great virtue but it requires hard work to achieve
it and education to appreciate it. And to make matters worse:
complexity sells better.}\newline
-- Edsger W. Dijkstra
\end{quote}

By making things harder than they need to be, companies can motivate
the sale  of intricate tools to clients who wished there was a
simpler way. Well, maybe there is.



\bibliography{ezr.bib}
\end{document}
